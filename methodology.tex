\section{Methodology}

For the development of our Coach Freelancing Platform, we wanted to choose the software development process that would suit the needs and time constraints of this project. To achieve this, we dedicated an entire meeting to discuss each process we have learned about and list their pros and cons and arrived at the conclusion that we wanted to adopt an agile methodology, since they are the best suited for Software Development. Agile utilize iterative and incremental development. It promotes flexible responses to change and aims to deliver working software frequently, with a preference for shorter time cycles. In software development requirements can change rapidly. Agile is particularly effective, because the project's direction can be changed quickly, but it requires a high level of coordination and communication among team members.
During our Software Engineering lectures we learned about many methodologies, here we will go through the most notable ones and argue why we chose to work with the agile framework: Scrum.

\subsection{Waterfall Model:}
This model is one of the earliest methodologies in engineering, it was and still is really popular outside of Software Engineering. It is based on linear sequential flow, meaning that any phase in the development process begins only after the previous one has been completed. The model is simple and easy to understand and use, which makes it suitable for projects with well-defined requirements that are unlikely to change, because after the report is done, and the project is planned out ahead, we would have to work according to that plan, even if user-requirements have not been correctly anticipated or have changed. The changing requirements typical in Software Development made this methodology unsuitable for us.

\subsection{Spiral Model:}
The Spiral model combines iterative development with the systematic aspects of the Waterfall model. It allows for incremental refinements through each iteration or spiral. Each spiral starts with a set of goals and ends with the client reviewing the progress, which helps in early detection and reduction of risks. This model is particularly useful for large, complex, and high-risk projects. However, it can be more costly than other methods due to the ongoing evaluation and risk analysis and additionally, the model's complexity necessitates strict adherence to protocols and increased documentation.

\subsection{V-Model:}
The V-Model, or Validation and Verification model, provides a more structured approach which entails a lengthier planning period analogous to the waterfall method of software development. The planning of it must be very reliable as all the successive development phases almost exclusively rely on the previous planning phase. This model is often used in safety critical systems, because is useful for catching defects in early stages, which makes it cheaper and less time-consuming to fix them compared to later in the development process. Given the importance of the ahead of time planning the framework results to be quite inflexible and doesn't take into consideration the possible changing of requirements during the development of the project.

\subsection{Unified Process}
This methodology is structured, iterative, and incremental and it emphasizes risk assessment, architecture, and continuous stakeholder involvement. It achieves these through distinct phases(Inception, Elaboration, Construction, Transition) and iterations which are, unlike Scrum, different length timeboxes, and not emphasized the same. Its values include adaptability, quality focus, and comprehensive documentation. While the Unified Process is suitable for complex, large-scale projects due to its detailed planning and risk mitigation, it can be expensive and complex, making it less agile and flexible compared to Scrum.

\subsection{Agile}
We already mentioned why we choose the agile methodology, now we had to choose a framework that utilizes it. We learned about many frameworks, a few of those were frameworks for large scale projects, which we aren't going to go in depth about here, since the magnitude of this project doesn't require it. We learned in depth about XP and Scrum. Since both of these are mostly similar, here we are going to mention a few key features and then argue why scrum is the more suitable for our project.
\begin{itemize}
    \item Extreme Programming\\ Extreme programming focuses on customer satisfaction and responding to changing customer requirements through practices like pair programming, test-driven development, and continuous integration. It also encourages frequent releases in short development cycles to improve productivity and introduce checkpoints for new customer requirements and focuses on simplicity, communication, feedback, and courage among the development team.
    \item Scrum \\ Scrum is built on the Agile Manifesto and the 12 Agile Principles. It uses fixed-length iterations called Sprints, lasting typically 2-4 weeks. The Scrum Team has defined roles: Product Owner, Scrum Master, Development Team. In and around the sprints there are structured events: Sprint Planning, Daily Scrum, Sprint Review, Sprint Retrospective. Scrum helps teams deliver value incrementally in a collaborative way. It has the artifacts Product Backlog, Sprint Backlog and Increment. It is important to maintain a prioritized Product Backlog to manage changing requirements and ensure that the team works on the most valuable tasks.
\end{itemize}

We chose scrum, because it is ideal for a small software startup as it fosters rapid development and adaptation to changing market demands. Scrum's lightweight framework is easy to understand and implement, allowing our team to stay focused, aligned, and responsive to our customers' feedback and business priorities. It promotes team collaboration and self-organization, and ensures regular delivery of valuable software increments, which values are important to almost all Software Developments.

\subsection{Epics and User Stories}
As part of Scrum we set up a Backlog in Atlassian's Jira webapplication. With the help of writing User Stories we pieced together Epics, which we are going to list below, and through those we identified issues and tasks, which all have been added to the backlog and can in the case of actuall production be put in different Sprints. \\
\textbf{Epics:}
\begin{itemize}

    \item Transition / Stripe integration
    \begin{itemize}
        \item Receive Payment
        \item Pay for Coaching
        \item Take a Cut From Payments
        \item Change Payment Information
        \item Change Billing Infromation
    \end{itemize}

    \item Explore Page
    \begin{itemize}
        \item Find Coach Through Explore Page
        \item Filter Coaches
        \item Get Found
        \item Control How to Appear on The Explore Page
        \item Get Notified if Coachin Spot is Freed
    \end{itemize}

    \item Chat
    \begin{itemize}
        \item Chatting with Coach
        \item Send and Receive Attachments
        \item Chatting with Trainee
        \item Seeing Trainee Details in a Sidebar
        \item Selecting Different Trainees to Chat to
    \end{itemize}

    \item Calling / Video-Calling
    \begin{itemize}
        \item Have Call between Coach and Trainee
        \item Have Videocall between Coach and Trainee
    \end{itemize}

    \item User Statistics
    \begin{itemize}
        \item See Personal Statistics
        \item See Statistics of Specific Trainee
        \item See Average Statistics of All Trainees
        \item Update / Log a Specific Metric
        \item See Workout Specific Statistics
        \item See Statistics for Selected Time Period
    \end{itemize}

    \item Workout Builder
    \begin{itemize}
        \item Create a New Workout
        \item Seva a Workout Template
        \item Use a Workout Template
        \item Pescribe a Workout
        \item Search for a Saved Workout
        \item Edit a Saved Workout
    \end{itemize}

\end{itemize}
\clearpage
