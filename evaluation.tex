

\section{Evaluation}

The LinkedGym Coach Freelancing Platform project has been an eye-opening and enlightening challenge, which has armed us with valuable insights into the software development process and project management. In this evaluation we aim to assess various aspects of the project, from the achievement of objectives to the performance of the team and the methodologies employed.

\subsection{Project Objectives and Goals Evaluation}
The objective of LinkedGym was to create a platform that elevates the online coaching experience by baking the communication services, custom plan creation, and progress tracking all into one convenient and intuitive application. This objective in our opinion can be met, according to the research, the succes of the prototyping and the other exercises we completed as part of the course. The app successfully addresses and solves the problem we identified with the user experience in current online coaching apps, providing a more cohesive and user-friendly interface.

\subsection{Project Management and Processes}
The project was planned to be managed using the Scrum framework, which will in theory will be effective given the iterative and flexible nature of the development process. If the development were to get under way, the team would adhere to regular sprint cycles, continuously improving and adaptating to the changing requirements.

\subsection{Team Performance}
The team demonstrated strong collaboration and communication throughout the project. Each member contributed to almost every aspect of the project, from initial brainstorming, design and diagrams to the business canvas. The use of clear, to the point communication has helped us quickly resolve any issue that came up. We used Trello to keep clear who has what objective, and it has been at great help.

\subsection{Lessons Learned}
Several lessons were learned throughout the project. The importance of thorough initial planning and clear requirement definitions was underscored, as these elements proved critical in guiding development efforts. Additionally, the value of iterative development and continuous user feedback was highlighted, enabling the team to refine the platform effectively.

\subsection{Overall Success and Areas for Improvement}
Overall, the LinkedGym project can be deemed a success, having met its primary objectives and delivered a functional prototype. However, there are areas for improvement, including enhancing the user interface, expanding the feature set, and securing additional funding for further development. Future iterations should focus on incorporating more extensive user feedback and exploring additional revenue models to ensure the platform's viability and growth.

\clearpage