\documentclass[12pt]{report}
\usepackage{amsmath}
\usepackage{graphicx}
\usepackage{hyperref}
\usepackage[utf8]{inputenc}
\title{Coach Freelancing Platform Project for Software Engineering Semester 2}
\author{Francesco Schenone \and Kacper Grzyb \and Levente Sohar \and Sebestyen Deak \and Ignat Bozhinov}
\date{2024-02-24}
\begin{document}
\maketitle
\section{Introduction}
\section{Software Process}
Chapter 2: Software Process

For the development of our Coach Freelancing Platform, we wanted to choose the software development process that would suit the needs and time constraints of this project. To achieve this, we dedicated an entire meeting to discuss each process we have learned about and list their pros and cons. We arrived at a conclusion that Scrum will be the most adequate. We also decided to break down the reasoning for that decision method by method.

Waterfall:
We chose Scrum over the Waterfall Method in the case of our Coach Freelancing Platform because it’s more scalable, more flexible and has the option to respond to user feedback. During the development process of Software Products, it is important to have the ability of changing a feature or interface. The single waterfall method doesn’t allow this kind of flexibility because it relies much more on the planning. After the report is done, and the project is planned out ahead, we would have to work according to that plan, even if user-requirements have not been correctly anticipated or have changed. 

V-Model: 
When comparing scrum to the V-model of software development, the V-model provides a more structured approach which entails a lengthier planning period analogous to the waterfall method of software development. The planning of it must be very reliable as all the successive development phases almost exclusively rely on the previous planning phase. Given the importance of the ahead of time planning the framework results to be quite inflexible and doesn’t take into consideration the possible changing of requirements during the development of the project.  Just from these attributes we can already steer away from this model of development as the requirements for our coaching-platform are not as defined as they would need to be and are very likely to for a successful application of the V-model. Additionally, our requirements are very likely to change posing a further challenge if we would use the v-model development method whilst in both cases using agile together with scrum would fit our needs very well.
The V-model development framework might be suitable for projects in which requirements are very unlikely to change and which require a very high level of stability and polishment once the product given the incorporation and heavy focus the V-model gives to testing. In fact, it provides for a testing phase after each development period which, ensures a more polished end-product but also slows down the overall development of the project. During the hypothetical development of our platform speed would be of the most utter importance even if it meant sacrificing some stability. Again, scrum is the better fit for us, as the development speed is faster, and it is always possible to include testing in the sprints. 

Spiral: 
After comparing scrum with the spiral iterative approach. However, major differences can be seen when compared directly.  When working on larger projects the spiral model can become complex due to the difficult task of having to coordinate multiple iterations and risk handling.  As it comes to time, scrum is superior. Going into the project, the number of required phases is often unknown, making time management almost impossible, leading to potential delays and budgetary overruns. Additionally, the model's complexity necessitates strict adherence to protocols and increased documentation. Moreover, customer interactions are fewer in comparison to those at the end of each sprint in the Scrum development process. In contrast, Scrum's streamlined approach proves superior in time efficiency, making it a more pragmatic choice for project management, particularly in larger-scale projects.
Peanut
Unified Process
When it comes to flexibility UP uses a more structured and descriptive method. This means that each member has a given task and they complete that. The project does not change by a significant degree throughout the stages of development. UP emphasises comprehensive documentation to capture requirements, design decisions, and testing activities, which may require additional resources for documentation and formal processes.
When it comes to the time section unified process allows a defined lifecycle with distinct phases. Compered to scrum it is does not focus on iterative phases therefore less time is required to go over the processes repeatedly.

With all of that considered we see that the Agile method Scrum is the clear solution to our problem. The flexibility provided and the sprint-based work system allows us to adjust our requirements and develop the project swiftly and efficiently. It also enables us to implement feedback given to us by our professors and evenly distribute workload. With Scrum we can become a self-organizing team that makes consistent progress throughout the course and fits within the deadlines given by the instructors. 

\end{document}