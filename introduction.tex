\section{Introduction}

\subsection{What is the issue we are trying to solve?}
More and more fitness influencer are trying to increase their income and reach by offering one-on-one coaching often accompanied by custom training- and diet plans. More generally, as consequence of the COVID-19 pandemic, an ever increasing number of traditional fitness coaches started to offer their coaching online as well.
We can say from first hand experience that the resulting coaching can become quite messy independent from the coach.
The user experience of the current influencer coaching is really bad, members of our team had first hand experience of how terrible it can be; it is very fragmented: communication channels are switched constantly, training and diet plans are hosted on seemingly random platforms and since it is all internet based, establishing initial trust / contact can result challenging. Our guess is that the GDPR is hardly respected.
That's why want to offer a unified solution for influencer and fitness coaching to remediate to all these problems.

\subsection{Is LinkedGym suitable for the market}

Based on our limited market analysis and other business analysis tools that we applied throughout the project we can confidently say that we do believe that LinkedGym could be a product in which consumers will have interest in, if launched. It won't grow by itself and will probably need to rely on influencer deals and need a decent marketing budget after launch which could be a limiting factor to its uptake.

\subsection{Could we produce LinkedGym}
In the scenario we would find ourselves in, if we were to concretely develop this product, most likely that the company developing it would be a startup with no- to very limited funding. With the abilities and knowledge we currently have, the successful development and deployment of LinkedGym would be very difficult, potentially possible if the motivation were high enough. It is to be noted that our abilities to develop LinkedGym, would probably be sufficient after the completion of the third semester (distributed interactive software systems) in our education.

\subsection{WHat is the basis of the project}

The only actual requirement for this project was that it had to have some kind of database as it was needed for the successful completion of all exercises given.
Various ideas surfaced during the brainstorming phase, and LinkedGym was not even our first choice. Ultimately LinkedGym turned out to be a better fir for two reasons: "founder fit" as all of the group members are fitness enthusiasts which amounts to a decent understanding of the current state of the fitness industry and it provided a better playing ground for the various software engineering exercises.

\subsection{Outline of the project}

\subsection{Business model of LinkedGym}

The primary source of revenue of LinkedGym will come from transactional commissions, where the platform takes a percentage of every transaction made through our platform. Furthermore we plan to have additional revenue streams by letting sporting brands promote their products on our platform.
The business model will be discussed in further detail in chapter 5.

\subsection{Product outline}

LynkedGym is an online coaching platform that offer a streamlined experienced for coaches and trainees alike. Unlike self branded solutions and platforms LinkedGym unifies everything involved for a quality fitness coaching experience, while still providing the possibility to advance ones personal brand. Our solution offers a simple and secure way to establish communication and subsequently to carry it on in a integrated platform. LinkedGym provides a straightforward way to create and assign custom plans, track progress and facilitate the whole coaching experience.

\subsection{Competing products}

Based on our limited market research we did not find a platform which would be a direct competitor to LinkedGym. There are similar apps that offer ways to track progress and provide custom workout plans, made among others by influencers but none offer the possibility to have direct coaching. Alternatively, there are many self branded coaching websites which re-direct a potential customer to instagram direct messages or similar social messaging platforms.

As we did not find any directly competing product, it may mean one of three things: There is a market to be captured and our platform solves a real need, or there is no need for such a product, hence the lack of a similar product. Ultimately it could be that our "googling" skills need serious improvement, reason why we did not find anything similar.

\subsection{Goals and risk of LinkedGym}

We defined goals and risks of LinkedGym by answering a list of predetermined questions that helped us to elucidate them.

\textbf{What is the best thing that could happen?}
\begin{itemize}
  \item Multiple fitness influencer with a moderate amount of traction switch to our platform
\end{itemize}
\textbf{What do we need to do to succeed?}
\begin{itemize}
  \item We need to attract influencer as for them to start using our platform
  \item We need to provide a straightforward and simple user experience
  \item We need to streamline the whole coaching process
\end{itemize}

\textbf{What is the worst to happen?}
\begin{itemize}
  \item We do not attract neither influencers nor coaches
  \item No one is interested in using our platform, there is no actual need for it
\end{itemize}

\textbf{If we fail...}
\begin{itemize}
  \item We did not provide enough differentiation relatively to similar platforms
  \item We did non manage to attract users
  \item  The profit margins of the platform turn out to be too low to be a viable product
\end{itemize}


\textbf{What are possible pitfalls}
\begin{itemize}
  \item We overcomplicate the platform
  \item The scope is to big for us to complete it
\end{itemize}
